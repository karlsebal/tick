%% Generated by Sphinx.
\def\sphinxdocclass{report}
\documentclass[letterpaper,10pt,english]{sphinxmanual}
\ifdefined\pdfpxdimen
   \let\sphinxpxdimen\pdfpxdimen\else\newdimen\sphinxpxdimen
\fi \sphinxpxdimen=.75bp\relax

\PassOptionsToPackage{warn}{textcomp}
\usepackage[utf8]{inputenc}
\ifdefined\DeclareUnicodeCharacter
% support both utf8 and utf8x syntaxes
\edef\sphinxdqmaybe{\ifdefined\DeclareUnicodeCharacterAsOptional\string"\fi}
  \DeclareUnicodeCharacter{\sphinxdqmaybe00A0}{\nobreakspace}
  \DeclareUnicodeCharacter{\sphinxdqmaybe2500}{\sphinxunichar{2500}}
  \DeclareUnicodeCharacter{\sphinxdqmaybe2502}{\sphinxunichar{2502}}
  \DeclareUnicodeCharacter{\sphinxdqmaybe2514}{\sphinxunichar{2514}}
  \DeclareUnicodeCharacter{\sphinxdqmaybe251C}{\sphinxunichar{251C}}
  \DeclareUnicodeCharacter{\sphinxdqmaybe2572}{\textbackslash}
\fi
\usepackage{cmap}
\usepackage[T1]{fontenc}
\usepackage{amsmath,amssymb,amstext}
\usepackage{babel}
\usepackage{times}
\usepackage[Bjarne]{fncychap}
\usepackage{sphinx}

\fvset{fontsize=\small}
\usepackage{geometry}

% Include hyperref last.
\usepackage{hyperref}
% Fix anchor placement for figures with captions.
\usepackage{hypcap}% it must be loaded after hyperref.
% Set up styles of URL: it should be placed after hyperref.
\urlstyle{same}

\addto\captionsenglish{\renewcommand{\figurename}{Fig.}}
\addto\captionsenglish{\renewcommand{\tablename}{Table}}
\addto\captionsenglish{\renewcommand{\literalblockname}{Listing}}

\addto\captionsenglish{\renewcommand{\literalblockcontinuedname}{continued from previous page}}
\addto\captionsenglish{\renewcommand{\literalblockcontinuesname}{continues on next page}}
\addto\captionsenglish{\renewcommand{\sphinxnonalphabeticalgroupname}{Non-alphabetical}}
\addto\captionsenglish{\renewcommand{\sphinxsymbolsname}{Symbols}}
\addto\captionsenglish{\renewcommand{\sphinxnumbersname}{Numbers}}

\addto\extrasenglish{\def\pageautorefname{page}}

\setcounter{tocdepth}{1}



\title{tick Documentation}
\date{Dec 23, 2018}
\release{0.2alpha}
\author{k}
\newcommand{\sphinxlogo}{\vbox{}}
\renewcommand{\releasename}{Release}
\makeindex
\begin{document}

\pagestyle{empty}
\maketitle
\pagestyle{plain}
\sphinxtableofcontents
\pagestyle{normal}
\phantomsection\label{\detokenize{index::doc}}


\sphinxtitleref{tick} is a command line tool keeping track of your working hours, holiday, over-/underhang which can be finally parsed into a nice and tidy \sphinxcode{\sphinxupquote{xlsx}} file.


\chapter{Contents}
\label{\detokenize{index:contents}}

\section{User’s Manual}
\label{\detokenize{usermanual:users-manual}}\label{\detokenize{usermanual::doc}}

\subsection{CLI}
\label{\detokenize{usermanual:cli}}

\subsubsection{Options}
\label{\detokenize{usermanual:options}}\index{command line option@\spxentry{command line option}!-h, --help@\spxentry{-h, --help}}\index{-h, --help@\spxentry{-h, --help}!command line option@\spxentry{command line option}}

\begin{fulllineitems}
\phantomsection\label{\detokenize{usermanual:cmdoption-h}}\pysigline{\sphinxbfcode{\sphinxupquote{-h}}\sphinxcode{\sphinxupquote{}}\sphinxcode{\sphinxupquote{,~}}\sphinxbfcode{\sphinxupquote{-{-}help}}\sphinxcode{\sphinxupquote{}}}
\end{fulllineitems}


help
\index{command line option@\spxentry{command line option}!-y, --year \textless{}year\textgreater{}@\spxentry{-y, --year \textless{}year\textgreater{}}}\index{-y, --year \textless{}year\textgreater{}@\spxentry{-y, --year \textless{}year\textgreater{}}!command line option@\spxentry{command line option}}

\begin{fulllineitems}
\phantomsection\label{\detokenize{usermanual:cmdoption-y}}\pysigline{\sphinxbfcode{\sphinxupquote{-y}}\sphinxcode{\sphinxupquote{}}\sphinxcode{\sphinxupquote{,~}}\sphinxbfcode{\sphinxupquote{-{-}year}}\sphinxcode{\sphinxupquote{~\textless{}year\textgreater{}}}}
\end{fulllineitems}


set year
\index{command line option@\spxentry{command line option}!-m, --month \textless{}month\textgreater{}@\spxentry{-m, --month \textless{}month\textgreater{}}}\index{-m, --month \textless{}month\textgreater{}@\spxentry{-m, --month \textless{}month\textgreater{}}!command line option@\spxentry{command line option}}

\begin{fulllineitems}
\phantomsection\label{\detokenize{usermanual:cmdoption-m}}\pysigline{\sphinxbfcode{\sphinxupquote{-m}}\sphinxcode{\sphinxupquote{}}\sphinxcode{\sphinxupquote{,~}}\sphinxbfcode{\sphinxupquote{-{-}month}}\sphinxcode{\sphinxupquote{~\textless{}month\textgreater{}}}}
\end{fulllineitems}


set month
\index{command line option@\spxentry{command line option}!-d, --day \textless{}dom\textgreater{}@\spxentry{-d, --day \textless{}dom\textgreater{}}}\index{-d, --day \textless{}dom\textgreater{}@\spxentry{-d, --day \textless{}dom\textgreater{}}!command line option@\spxentry{command line option}}

\begin{fulllineitems}
\phantomsection\label{\detokenize{usermanual:cmdoption-d}}\pysigline{\sphinxbfcode{\sphinxupquote{-d}}\sphinxcode{\sphinxupquote{}}\sphinxcode{\sphinxupquote{,~}}\sphinxbfcode{\sphinxupquote{-{-}day}}\sphinxcode{\sphinxupquote{~\textless{}dom\textgreater{}}}}
\end{fulllineitems}


set day of month
\index{command line option@\spxentry{command line option}!-D, --date@\spxentry{-D, --date}}\index{-D, --date@\spxentry{-D, --date}!command line option@\spxentry{command line option}}

\begin{fulllineitems}
\phantomsection\label{\detokenize{usermanual:cmdoption-date}}\pysigline{\sphinxbfcode{\sphinxupquote{-D}}\sphinxcode{\sphinxupquote{}}\sphinxcode{\sphinxupquote{,~}}\sphinxbfcode{\sphinxupquote{-{-}date}}\sphinxcode{\sphinxupquote{}}}
\end{fulllineitems}


set date as if set via \sphinxcode{\sphinxupquote{date -d}}
\index{command line option@\spxentry{command line option}!-Y, --yesterday@\spxentry{-Y, --yesterday}}\index{-Y, --yesterday@\spxentry{-Y, --yesterday}!command line option@\spxentry{command line option}}

\begin{fulllineitems}
\phantomsection\label{\detokenize{usermanual:cmdoption-yesterday}}\pysigline{\sphinxbfcode{\sphinxupquote{-Y}}\sphinxcode{\sphinxupquote{}}\sphinxcode{\sphinxupquote{,~}}\sphinxbfcode{\sphinxupquote{-{-}yesterday}}\sphinxcode{\sphinxupquote{}}}
\end{fulllineitems}


set date to yesterday

\begin{sphinxadmonition}{note}{Note:}
The date given via {\hyperref[\detokenize{usermanual:cmdoption-date}]{\sphinxcrossref{\sphinxcode{\sphinxupquote{-D}}}}} is parsed directly into \sphinxstyleliteralstrong{\sphinxupquote{date -d}}.
\end{sphinxadmonition}


\subsubsection{Commands}
\label{\detokenize{usermanual:commands}}\begin{description}
\item[{\textless{}HH{[}:MM{]}\textbar{}:MM\textgreater{} \textless{}description\textgreater{}}] \leavevmode
appends a work \textless{}description\textgreater{} with the duration of \textless{}number\textgreater{}. Duration is given either in hours or
hours and minutes separated by colon or minutes preceded by a colon.

\item[{\textless{}HH:MM\textgreater{}-\textless{}HH:MM\textgreater{} \textless{}description\textgreater{}}] \leavevmode
appends a work \textless{}description\textgreater{} for the period given.

\item[{parse}] \leavevmode
parse the protocol related to the month set

The file named \sphinxcode{\sphinxupquote{Arbeitszeiten\_\textless{}YY-MM\textgreater{}.xlsx}} will be placed in the working directory

\item[{report}] \leavevmode
parse the protocol related to the month set and send the \sphinxtitleref{xlsx} file to a configured mail address

\item[{undo}] \leavevmode
remove the last entry in the month’s protocol to which the date is set

\end{description}


\subsection{Configuration}
\label{\detokenize{usermanual:configuration}}
As with version 0.2alpha all configuration is hard coded. Changes are to be made in the shell script. See {\hyperref[\detokenize{devmanual::doc}]{\sphinxcrossref{\DUrole{doc}{Developer’s Manual}}}} for details.


\section{Developer’s Manual}
\label{\detokenize{devmanual:developers-manual}}\label{\detokenize{devmanual::doc}}

\subsection{Overview}
\label{\detokenize{devmanual:overview}}
There are two components doing the job:
\begin{itemize}
\item {} 
the shell script as controller, making entries, calling the parser and sending reports.

\item {} 
the parser creating the \sphinxtitleref{xlsx} files

\end{itemize}


\subsection{files and formats}
\label{\detokenize{devmanual:files-and-formats}}
The controller files are stored in \sphinxcode{\sphinxupquote{\textasciitilde{}/.tick/}} where a \sphinxcode{\sphinxupquote{protocol.csv}} is created.


\subsubsection{csv format}
\label{\detokenize{devmanual:csv-format}}
The csv follows the \sphinxstylestrong{standard format} of \sphinxtitleref{comma separated values} %
\begin{footnote}[1]\sphinxAtStartFootnote
\sphinxurl{https://en.wikipedia.org/wiki/Comma-separated\_values}
%
\end{footnote}.
\begin{itemize}
\item {} 
tag

\item {} 
year

\item {} 
month

\item {} 
day

\item {} 
duration in seconds

\item {} 
from unixtime

\item {} 
to unixtime

\item {} 
description

\end{itemize}

It is not neccessary that both duration and from/to are set but if they are those two times spans are validated against each other.


\paragraph{available tags}
\label{\detokenize{devmanual:available-tags}}
Tags available in version 0.2alpha are:


\begin{savenotes}\sphinxattablestart
\centering
\begin{tabulary}{\linewidth}[t]{|T|T|}
\hline
\sphinxstyletheadfamily 
tag string
&\sphinxstyletheadfamily 
meaning
\\
\hline
e
&
Entry
\\
\hline
h
&
Holiday
\\
\hline
c
&
Overtime Compensation
\\
\hline
i
&
Illness
\\
\hline
o
&
Offset
\\
\hline
\end{tabulary}
\par
\sphinxattableend\end{savenotes}

\sphinxtitleref{offset} is used to add working hours to account.
When \sphinxtitleref{holiday} is used with duration, holidays are added.


\subsubsection{xlsx scheme}
\label{\detokenize{devmanual:xlsx-scheme}}
The \sphinxtitleref{excel} file consists of roughly 3 Parts:
\begin{itemize}
\item {} \begin{description}
\item[{Header with}] \leavevmode\begin{itemize}
\item {} 
Name

\item {} 
Month

\item {} 
Overview

\end{itemize}

\end{description}

\item {} 
Body

\item {} \begin{description}
\item[{Footer with}] \leavevmode\begin{itemize}
\item {} 
legend

\item {} 
infoline containing technical informations like parsing date and so on

\end{itemize}

\end{description}

\end{itemize}


\subsubsection{Legacy Formats}
\label{\detokenize{devmanual:legacy-formats}}
The predecessor \sphinxstyleliteralstrong{\sphinxupquote{etime}} used two files:
\begin{itemize}
\item {} 
the tracker put the data of a month into \sphinxcode{\sphinxupquote{\textasciitilde{}/.etime/\textless{}YY-MM.elog}} whith each line in the form \sphinxcode{\sphinxupquote{\textless{}DD\textgreater{} \textless{}duration in hours in float\textgreater{} \textless{}description\textgreater{}}}

\item {} \begin{description}
\item[{when parsing two files where created:}] \leavevmode\begin{itemize}
\item {} 
a \sphinxcode{\sphinxupquote{csv}} with each line of the form \sphinxcode{\sphinxupquote{\textless{}DD\textgreater{}.\textless{}MM\textgreater{}.;\textless{}Duration in float, comma as separator\textgreater{};"\textless{}description\textgreater{}"}}

\item {} 
a \sphinxcode{\sphinxupquote{xlsx}} containing a bare, unsorted list with full dates

\end{itemize}

\end{description}

\end{itemize}


\subsection{configuration}
\label{\detokenize{devmanual:configuration}}
Configuration is hard coded. Change the email address the report is sent to in the shell script.


\subsection{Components}
\label{\detokenize{devmanual:components}}

\subsubsection{Controller}
\label{\detokenize{devmanual:controller}}

\subsubsection{Parser}
\label{\detokenize{devmanual:parser}}

\subsection{API}
\label{\detokenize{devmanual:api}}

\subsubsection{Parser}
\label{\detokenize{devmanual:id2}}

\paragraph{parser}
\label{\detokenize{devmanual:id3}}\phantomsection\label{\detokenize{devmanual:module-parser}}\index{parser (module)@\spxentry{parser}\spxextra{module}}
parser for \sphinxstyleliteralstrong{\sphinxupquote{tick}}
\index{parse\_csv\_protocol() (in module parser)@\spxentry{parse\_csv\_protocol()}\spxextra{in module parser}}

\begin{fulllineitems}
\phantomsection\label{\detokenize{devmanual:parser.parse_csv_protocol}}\pysiglinewithargsret{\sphinxcode{\sphinxupquote{parser.}}\sphinxbfcode{\sphinxupquote{parse\_csv\_protocol}}}{\emph{protocol}}{}
parse a list of \sphinxtitleref{csv} protocol entries into year. return year.
\begin{quote}\begin{description}
\item[{Parameters}] \leavevmode
\sphinxstyleliteralstrong{\sphinxupquote{protocol}} (\sphinxhref{https://docs.python.org/3/library/typing.html\#typing.Union}{\sphinxcode{\sphinxupquote{Union}}}{[}\sphinxhref{https://docs.python.org/3/library/stdtypes.html\#list}{\sphinxcode{\sphinxupquote{list}}}, \sphinxhref{https://docs.python.org/3/library/stdtypes.html\#tuple}{\sphinxcode{\sphinxupquote{tuple}}}{]}) -- protocol to parse

\item[{Return type}] \leavevmode
\sphinxhref{https://docs.python.org/3/library/stdtypes.html\#dict}{\sphinxcode{\sphinxupquote{dict}}}

\end{description}\end{quote}

\end{fulllineitems}



\paragraph{protocol}
\label{\detokenize{devmanual:module-protocol}}\label{\detokenize{devmanual:protocol}}\index{protocol (module)@\spxentry{protocol}\spxextra{module}}
This module provides the Month and Year classes
\index{ConfusingDataException@\spxentry{ConfusingDataException}}

\begin{fulllineitems}
\phantomsection\label{\detokenize{devmanual:protocol.ConfusingDataException}}\pysigline{\sphinxbfcode{\sphinxupquote{exception }}\sphinxcode{\sphinxupquote{protocol.}}\sphinxbfcode{\sphinxupquote{ConfusingDataException}}}
\end{fulllineitems}

\index{InvalidDateException@\spxentry{InvalidDateException}}

\begin{fulllineitems}
\phantomsection\label{\detokenize{devmanual:protocol.InvalidDateException}}\pysigline{\sphinxbfcode{\sphinxupquote{exception }}\sphinxcode{\sphinxupquote{protocol.}}\sphinxbfcode{\sphinxupquote{InvalidDateException}}}
\end{fulllineitems}

\index{Month (class in protocol)@\spxentry{Month}\spxextra{class in protocol}}

\begin{fulllineitems}
\phantomsection\label{\detokenize{devmanual:protocol.Month}}\pysiglinewithargsret{\sphinxbfcode{\sphinxupquote{class }}\sphinxcode{\sphinxupquote{protocol.}}\sphinxbfcode{\sphinxupquote{Month}}}{\emph{year=0}, \emph{month=0}, \emph{holidays\_left=0}, \emph{working\_hours\_account=0}, \emph{hours\_worth\_working\_day=4}}{}
Provides the work time protocol for one month
\begin{quote}\begin{description}
\item[{Parameters}] \leavevmode\begin{itemize}
\item {} 
\sphinxstyleliteralstrong{\sphinxupquote{holidays\_left}} (\sphinxhref{https://docs.python.org/3/library/functions.html\#int}{\sphinxcode{\sphinxupquote{int}}}) -- number of holidays left, 0 if omitted

\item {} 
\sphinxstyleliteralstrong{\sphinxupquote{working\_hours\_account}} (\sphinxhref{https://docs.python.org/3/library/functions.html\#int}{\sphinxcode{\sphinxupquote{int}}}) -- credit of working hours in \sphinxstyleemphasis{seconds}, 0 if omitted

\item {} 
\sphinxstyleliteralstrong{\sphinxupquote{yearmonth}} -- integer of the form {[}{[}YY{]}YY{]}MM. Current time if omitted. Current year if only month is given. 20th century if first two digits are missing.

\item {} 
\sphinxstyleliteralstrong{\sphinxupquote{hours\_worth\_working\_day}} (\sphinxhref{https://docs.python.org/3/library/functions.html\#int}{\sphinxcode{\sphinxupquote{int}}}) -- number of hours a working day is worth
If only one digit is given it will be padded with a leading 0.

\end{itemize}

\item[{Raises}] \leavevmode
{\hyperref[\detokenize{devmanual:protocol.InvalidDateException}]{\sphinxcrossref{\sphinxstyleliteralstrong{\sphinxupquote{InvalidDateException}}}}} -- raised when the Date given is nonsense.

\end{description}\end{quote}
\index{append() (protocol.Month method)@\spxentry{append()}\spxextra{protocol.Month method}}

\begin{fulllineitems}
\phantomsection\label{\detokenize{devmanual:protocol.Month.append}}\pysiglinewithargsret{\sphinxbfcode{\sphinxupquote{append}}}{\emph{tag}, \emph{day}, \emph{duration=0}, \emph{from\_unixtime=0}, \emph{to\_unixtime=0}, \emph{description=None}}{}
append an entry to the protocol
\begin{quote}\begin{description}
\item[{Parameters}] \leavevmode\begin{itemize}
\item {} 
\sphinxstyleliteralstrong{\sphinxupquote{tag}} (\sphinxhref{https://docs.python.org/3/library/stdtypes.html\#str}{\sphinxcode{\sphinxupquote{str}}}) -- tag

\item {} 
\sphinxstyleliteralstrong{\sphinxupquote{duration}} (\sphinxhref{https://docs.python.org/3/library/functions.html\#int}{\sphinxcode{\sphinxupquote{int}}}) -- duration of work in seconds

\item {} 
\sphinxstyleliteralstrong{\sphinxupquote{from\_unixtime}} (\sphinxhref{https://docs.python.org/3/library/functions.html\#int}{\sphinxcode{\sphinxupquote{int}}}) -- beginning of work in epoch

\item {} 
\sphinxstyleliteralstrong{\sphinxupquote{to\_unixtime}} (\sphinxhref{https://docs.python.org/3/library/functions.html\#int}{\sphinxcode{\sphinxupquote{int}}}) -- ending of work in epoch

\item {} 
\sphinxstyleliteralstrong{\sphinxupquote{description}} (\sphinxhref{https://docs.python.org/3/library/typing.html\#typing.Optional}{\sphinxcode{\sphinxupquote{Optional}}}{[}\sphinxhref{https://docs.python.org/3/library/stdtypes.html\#str}{\sphinxcode{\sphinxupquote{str}}}{]}) -- description

\end{itemize}

\item[{Raises}] \leavevmode
\sphinxstyleliteralstrong{\sphinxupquote{ConfusingData}} -- when duration and from/to do are both given and do not match

\item[{Return type}] \leavevmode
{\hyperref[\detokenize{devmanual:protocol.Month}]{\sphinxcrossref{\sphinxcode{\sphinxupquote{Month}}}}}

\end{description}\end{quote}

\end{fulllineitems}

\index{append\_protocol() (protocol.Month method)@\spxentry{append\_protocol()}\spxextra{protocol.Month method}}

\begin{fulllineitems}
\phantomsection\label{\detokenize{devmanual:protocol.Month.append_protocol}}\pysiglinewithargsret{\sphinxbfcode{\sphinxupquote{append\_protocol}}}{\emph{protocol}}{}
add a list or tuple of entries to the protocol
\begin{quote}\begin{description}
\item[{Return type}] \leavevmode
{\hyperref[\detokenize{devmanual:protocol.Month}]{\sphinxcrossref{\sphinxcode{\sphinxupquote{Month}}}}}

\end{description}\end{quote}

\end{fulllineitems}

\index{dump() (protocol.Month method)@\spxentry{dump()}\spxextra{protocol.Month method}}

\begin{fulllineitems}
\phantomsection\label{\detokenize{devmanual:protocol.Month.dump}}\pysiglinewithargsret{\sphinxbfcode{\sphinxupquote{dump}}}{}{}
return a dict with all values
\begin{quote}\begin{description}
\item[{Return type}] \leavevmode
\sphinxhref{https://docs.python.org/3/library/stdtypes.html\#dict}{\sphinxcode{\sphinxupquote{dict}}}

\end{description}\end{quote}

\end{fulllineitems}

\index{get\_next() (protocol.Month method)@\spxentry{get\_next()}\spxextra{protocol.Month method}}

\begin{fulllineitems}
\phantomsection\label{\detokenize{devmanual:protocol.Month.get_next}}\pysiglinewithargsret{\sphinxbfcode{\sphinxupquote{get\_next}}}{\emph{year=None}, \emph{month=None}}{}
return the next Month derived from the current

holidays left are transferred, working hours account
is adjusted.
\begin{quote}\begin{description}
\item[{Parameters}] \leavevmode
\sphinxstyleliteralstrong{\sphinxupquote{month}} -- you can give a different month. If omitted next
month is assumed.

\item[{Return type}] \leavevmode
{\hyperref[\detokenize{devmanual:protocol.Month}]{\sphinxcrossref{\sphinxcode{\sphinxupquote{Month}}}}}

\end{description}\end{quote}

\end{fulllineitems}

\index{pretty() (protocol.Month method)@\spxentry{pretty()}\spxextra{protocol.Month method}}

\begin{fulllineitems}
\phantomsection\label{\detokenize{devmanual:protocol.Month.pretty}}\pysiglinewithargsret{\sphinxbfcode{\sphinxupquote{pretty}}}{}{}
return object as pretty string
\begin{quote}\begin{description}
\item[{Return type}] \leavevmode
\sphinxhref{https://docs.python.org/3/library/stdtypes.html\#str}{\sphinxcode{\sphinxupquote{str}}}

\end{description}\end{quote}

\end{fulllineitems}


\end{fulllineitems}

\index{Season (class in protocol)@\spxentry{Season}\spxextra{class in protocol}}

\begin{fulllineitems}
\phantomsection\label{\detokenize{devmanual:protocol.Season}}\pysiglinewithargsret{\sphinxbfcode{\sphinxupquote{class }}\sphinxcode{\sphinxupquote{protocol.}}\sphinxbfcode{\sphinxupquote{Season}}}{\emph{t=0}, \emph{working\_hours\_account=0}}{}
contains a valid chain of Months
\index{add\_month() (protocol.Season method)@\spxentry{add\_month()}\spxextra{protocol.Season method}}

\begin{fulllineitems}
\phantomsection\label{\detokenize{devmanual:protocol.Season.add_month}}\pysiglinewithargsret{\sphinxbfcode{\sphinxupquote{add\_month}}}{\emph{month}}{}
append a protocol to the chain
\begin{quote}\begin{description}
\item[{Parameters}] \leavevmode
\sphinxstyleliteralstrong{\sphinxupquote{protocol}} -- protocol to add.

\item[{Raises}] \leavevmode
\sphinxhref{https://docs.python.org/3/library/exceptions.html\#ValueError}{\sphinxstyleliteralstrong{\sphinxupquote{ValueError}}} -- when validation fails

\item[{Return type}] \leavevmode
{\hyperref[\detokenize{devmanual:protocol.Season}]{\sphinxcrossref{\sphinxcode{\sphinxupquote{Season}}}}}

\end{description}\end{quote}

\end{fulllineitems}


\end{fulllineitems}



\section{Changelog}
\label{\detokenize{changelog:changelog}}\label{\detokenize{changelog::doc}}
\DUrole{versionmodified}{New in version 0.1: }Initial Version.


\chapter{Indices and tables}
\label{\detokenize{index:indices-and-tables}}\begin{itemize}
\item {} 
\DUrole{xref,std,std-ref}{genindex}

\item {} 
\DUrole{xref,std,std-ref}{modindex}

\item {} 
\DUrole{xref,std,std-ref}{search}

\end{itemize}


\renewcommand{\indexname}{Python Module Index}
\begin{sphinxtheindex}
\let\bigletter\sphinxstyleindexlettergroup
\bigletter{p}
\item\relax\sphinxstyleindexentry{parser}\sphinxstyleindexpageref{devmanual:\detokenize{module-parser}}
\item\relax\sphinxstyleindexentry{protocol}\sphinxstyleindexpageref{devmanual:\detokenize{module-protocol}}
\end{sphinxtheindex}

\renewcommand{\indexname}{Index}
\printindex
\end{document}