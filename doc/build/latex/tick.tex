%% Generated by Sphinx.
\def\sphinxdocclass{report}
\documentclass[letterpaper,10pt,english]{sphinxmanual}
\ifdefined\pdfpxdimen
   \let\sphinxpxdimen\pdfpxdimen\else\newdimen\sphinxpxdimen
\fi \sphinxpxdimen=.75bp\relax

\usepackage[utf8]{inputenc}
\ifdefined\DeclareUnicodeCharacter
 \ifdefined\DeclareUnicodeCharacterAsOptional
  \DeclareUnicodeCharacter{"00A0}{\nobreakspace}
  \DeclareUnicodeCharacter{"2500}{\sphinxunichar{2500}}
  \DeclareUnicodeCharacter{"2502}{\sphinxunichar{2502}}
  \DeclareUnicodeCharacter{"2514}{\sphinxunichar{2514}}
  \DeclareUnicodeCharacter{"251C}{\sphinxunichar{251C}}
  \DeclareUnicodeCharacter{"2572}{\textbackslash}
 \else
  \DeclareUnicodeCharacter{00A0}{\nobreakspace}
  \DeclareUnicodeCharacter{2500}{\sphinxunichar{2500}}
  \DeclareUnicodeCharacter{2502}{\sphinxunichar{2502}}
  \DeclareUnicodeCharacter{2514}{\sphinxunichar{2514}}
  \DeclareUnicodeCharacter{251C}{\sphinxunichar{251C}}
  \DeclareUnicodeCharacter{2572}{\textbackslash}
 \fi
\fi
\usepackage{cmap}
\usepackage[T1]{fontenc}
\usepackage{amsmath,amssymb,amstext}
\usepackage{babel}
\usepackage{times}
\usepackage[Bjarne]{fncychap}
\usepackage[dontkeepoldnames]{sphinx}

\usepackage{geometry}

% Include hyperref last.
\usepackage{hyperref}
% Fix anchor placement for figures with captions.
\usepackage{hypcap}% it must be loaded after hyperref.
% Set up styles of URL: it should be placed after hyperref.
\urlstyle{same}

\addto\captionsenglish{\renewcommand{\figurename}{Fig.}}
\addto\captionsenglish{\renewcommand{\tablename}{Table}}
\addto\captionsenglish{\renewcommand{\literalblockname}{Listing}}

\addto\captionsenglish{\renewcommand{\literalblockcontinuedname}{continued from previous page}}
\addto\captionsenglish{\renewcommand{\literalblockcontinuesname}{continues on next page}}

\addto\extrasenglish{\def\pageautorefname{page}}

\setcounter{tocdepth}{1}



\title{tick Documentation}
\date{Dec 05, 2017}
\release{0.1a}
\author{k}
\newcommand{\sphinxlogo}{\vbox{}}
\renewcommand{\releasename}{Release}
\makeindex

\begin{document}

\maketitle
\sphinxtableofcontents
\phantomsection\label{\detokenize{index::doc}}


\sphinxtitleref{tick} is a command line tool keeping track of your working hours, holiday, over-/underhang which can be finally parsed into a nice and tidy \sphinxcode{xlsx} file.


\chapter{Contents}
\label{\detokenize{index:time-clock}}\label{\detokenize{index:contents}}

\section{User’s Manual}
\label{\detokenize{usermanual::doc}}\label{\detokenize{usermanual:users-manual}}

\subsection{CLI}
\label{\detokenize{usermanual:cli}}

\subsubsection{Options}
\label{\detokenize{usermanual:options}}\begin{optionlist}{3cm}
\item [-h, -{-}help]  
help
\item [-y, -{-}year]  
set year
\item [-m, -{-}month]  
set month
\item [-d, -{-}day]  
set day
\item [-D, -{-}date]  
set date
\item [-Y, -{-}yesterday]  
set date to yesterday
\item [-L, -{-}lastmonth]  
set month to \sphinxcode{last month}
\end{optionlist}

\begin{sphinxadmonition}{note}{Note:}
The date given via \sphinxcode{-D} is parsed directly into \sphinxstyleliteralstrong{date}. So simulating the option \sphinxcode{-Y} via \sphinxcode{-D yesterday} is absolutely possible.
\end{sphinxadmonition}


\subsubsection{Commands}
\label{\detokenize{usermanual:commands}}\begin{description}
\item[{\textless{}number\textgreater{} \textless{}description\textgreater{}}] \leavevmode
appends a work \textless{}description\textgreater{} with the duration of \textless{}number\textgreater{}

\item[{\textless{}from\textgreater{}-\textless{}to\textgreater{} \textless{}description\textgreater{}}] \leavevmode
appends a work \textless{}description\textgreater{} in the period from \textless{}from\textgreater{} to \textless{}to\textgreater{} where both are \sphinxstyleliteralstrong{date} valid time strings.

\item[{parse}] \leavevmode
parse the protocol related to the month set

The file named \sphinxcode{Arbeitszeiten\_\textless{}YY-MM\textgreater{}.xlsx} will be placed in the working directory

\item[{report}] \leavevmode
parse the protocol related to the month set and send the \sphinxtitleref{xlsx} file to a configured mail address

\item[{undo}] \leavevmode
remove the last entry in the month’s protocol to which the date is set

\end{description}


\subsection{Configuration}
\label{\detokenize{usermanual:configuration}}
As with version 0.1a all configuration is hard coded. Changes are to be made in the shell script. See {\hyperref[\detokenize{devmanual::doc}]{\sphinxcrossref{\DUrole{doc}{Developer’s Manual}}}} for details.


\section{Developer’s Manual}
\label{\detokenize{devmanual::doc}}\label{\detokenize{devmanual:developers-manual}}

\subsection{Overview}
\label{\detokenize{devmanual:overview}}
There are two components doing the job:
\begin{itemize}
\item {} 
the shell script as controller, making entries, calling the parser and sending reports.

\item {} 
the parser creating the \sphinxtitleref{xlsx} files

\end{itemize}


\subsection{files and formats}
\label{\detokenize{devmanual:files-and-formats}}
The controller files are stored in \sphinxcode{\textasciitilde{}/.tick/} where a \sphinxcode{protocol.csv} is created.


\subsubsection{csv format}
\label{\detokenize{devmanual:csv-format}}
The csv follows the \sphinxstylestrong{standard format} of \sphinxtitleref{comma separated values} %
\begin{footnote}[1]\sphinxAtStartFootnote
\sphinxurl{https://en.wikipedia.org/wiki/Comma-separated\_values}
%
\end{footnote}.
\begin{itemize}
\item {} 
tag

\item {} 
year

\item {} 
month

\item {} 
day

\item {} 
duration in seconds

\item {} 
from unixtime

\item {} 
to unixtime

\item {} 
description

\end{itemize}

It is not neccessary that both duration and from/to are set but if they are those two times spans are validated against each other.


\paragraph{available tags}
\label{\detokenize{devmanual:available-tags}}
Tags available in version 0.1a are:


\begin{savenotes}\sphinxattablestart
\centering
\begin{tabulary}{\linewidth}[t]{|T|T|}
\hline
\sphinxstylethead{\sphinxstyletheadfamily 
tag string
\unskip}\relax &\sphinxstylethead{\sphinxstyletheadfamily 
meaning
\unskip}\relax \\
\hline
\textasciitilde{}
&
Comment
\\
\hline
e
&
Entry
\\
\hline
h
&
Holiday
\\
\hline
c
&
Overtime Compensation
\\
\hline
i
&
Illness
\\
\hline
o
&
Offset
\\
\hline
\end{tabulary}
\par
\sphinxattableend\end{savenotes}

\sphinxtitleref{offset} is used to add holidays or working hours to account.


\subsubsection{xlsx scheme}
\label{\detokenize{devmanual:xlsx-scheme}}
The \sphinxtitleref{excel} file consists of roughly 3 Parts:
\begin{itemize}
\item {} \begin{description}
\item[{Header with}] \leavevmode\begin{itemize}
\item {} 
Name

\item {} 
Month

\item {} 
Overview

\end{itemize}

\end{description}

\item {} 
Body

\item {} \begin{description}
\item[{Footer with}] \leavevmode\begin{itemize}
\item {} 
legend

\item {} 
infoline containing technical informations like parsing date and so on

\end{itemize}

\end{description}

\end{itemize}


\subsubsection{Legacy Formats}
\label{\detokenize{devmanual:legacy-formats}}
The predecessor \sphinxstyleliteralstrong{etime} used two files:
\begin{itemize}
\item {} 
the tracker put the data of a month into \sphinxcode{\textasciitilde{}/.etime/\textless{}YY-MM.elog} whith each line in the form \sphinxcode{\textless{}DD\textgreater{} \textless{}duration in hours in float\textgreater{} \textless{}description\textgreater{}}

\item {} \begin{description}
\item[{when parsing two files where created:}] \leavevmode\begin{itemize}
\item {} 
a \sphinxcode{csv} with each line of the form \sphinxcode{\textless{}DD\textgreater{}.\textless{}MM\textgreater{}.;\textless{}Duration in float, comma as separator\textgreater{};"\textless{}description\textgreater{}"}

\item {} 
a \sphinxcode{xlsx} containing a bare, unsorted list with full dates

\end{itemize}

\end{description}

\end{itemize}


\subsection{configuration}
\label{\detokenize{devmanual:configuration}}
Configuration is hard coded. Change the email address the report is sent to in the shell script.


\subsection{Components}
\label{\detokenize{devmanual:components}}

\subsubsection{Controller}
\label{\detokenize{devmanual:controller}}

\subsubsection{Parser}
\label{\detokenize{devmanual:parser}}

\subsection{API}
\label{\detokenize{devmanual:api}}

\subsubsection{Parser}
\label{\detokenize{devmanual:id2}}

\paragraph{parser}
\label{\detokenize{devmanual:id3}}\phantomsection\label{\detokenize{devmanual:module-parser}}\index{parser (module)}
parser for \sphinxstyleliteralstrong{tick}
\index{parse\_csv\_protocol() (in module parser)}

\begin{fulllineitems}
\phantomsection\label{\detokenize{devmanual:parser.parse_csv_protocol}}\pysiglinewithargsret{\sphinxcode{parser.}\sphinxbfcode{parse\_csv\_protocol}}{\emph{protocol}}{}
parse a list of \sphinxtitleref{csv} protocol entries into year. return year.
\begin{quote}\begin{description}
\item[{Parameters}] \leavevmode
\sphinxstyleliteralstrong{protocol} (\sphinxcode{Union}{[}\sphinxcode{list}, \sphinxcode{tuple}{]}) \textendash{} protocol to parse

\end{description}\end{quote}

\begin{sphinxadmonition}{warning}{Warning:}
missing months are not padded. The {\hyperref[\detokenize{devmanual:protocol.Month}]{\sphinxcrossref{\sphinxcode{protocol.Month}}}}
resulting from the {\hyperref[\detokenize{devmanual:protocol.Month.get_next}]{\sphinxcrossref{\sphinxcode{protocol.Month.get\_next()}}}} method 
is adjusted to reflect the next month in chain
\end{sphinxadmonition}
\begin{quote}\begin{description}
\item[{Return type}] \leavevmode
\sphinxhref{https://docs.python.org/2/library/stdtypes.html\#dict}{\sphinxcode{dict}}

\end{description}\end{quote}

\end{fulllineitems}



\paragraph{protocol}
\label{\detokenize{devmanual:module-protocol}}\label{\detokenize{devmanual:protocol}}\index{protocol (module)}
This module provides the Month and Year classes
\index{Month (class in protocol)}

\begin{fulllineitems}
\phantomsection\label{\detokenize{devmanual:protocol.Month}}\pysiglinewithargsret{\sphinxbfcode{class }\sphinxcode{protocol.}\sphinxbfcode{Month}}{\emph{year: int = 0}, \emph{month: int = 0}, \emph{holidays\_left: int = 0}, \emph{working\_hours\_account: int = 0}, \emph{hours\_worth\_working\_day: int = 4}}{}
Provides the work time protocol for one month
\begin{quote}\begin{description}
\item[{Parameters}] \leavevmode\begin{itemize}
\item {} 
\sphinxstyleliteralstrong{holidays\_left} (\sphinxhref{https://docs.python.org/2/library/functions.html\#int}{\sphinxcode{int}}) \textendash{} number of holidays left, 0 if omitted

\item {} 
\sphinxstyleliteralstrong{working\_hours\_account} (\sphinxhref{https://docs.python.org/2/library/functions.html\#int}{\sphinxcode{int}}) \textendash{} credit of working hours in \sphinxstyleemphasis{seconds}, 0 if omitted

\item {} 
\sphinxstyleliteralstrong{yearmonth} \textendash{} integer of the form {[}{[}YY{]}YY{]}MM. Current time if omitted. Current year if only month is given. 20th century if first two digits are missing.

\item {} 
\sphinxstyleliteralstrong{hours\_worth\_working\_day} (\sphinxhref{https://docs.python.org/2/library/functions.html\#int}{\sphinxcode{int}}) \textendash{} number of hours a working day is worth
If only one digit is given it will be padded with a leading 0.

\end{itemize}

\item[{Raises}] \leavevmode
\sphinxstyleliteralstrong{InvalidDateException} \textendash{} raised when the Date given is nonsense.

\end{description}\end{quote}
\index{append() (protocol.Month method)}

\begin{fulllineitems}
\phantomsection\label{\detokenize{devmanual:protocol.Month.append}}\pysiglinewithargsret{\sphinxbfcode{append}}{\emph{tag}, \emph{day}, \emph{duration=0}, \emph{from\_unixtime=0}, \emph{to\_unixtime=0}, \emph{description=None}}{}
append an entry to the protocol
\begin{quote}\begin{description}
\item[{Parameters}] \leavevmode\begin{itemize}
\item {} 
\sphinxstyleliteralstrong{tag} (\sphinxhref{https://docs.python.org/2/library/functions.html\#str}{\sphinxcode{str}}) \textendash{} tag

\item {} 
\sphinxstyleliteralstrong{duration} (\sphinxhref{https://docs.python.org/2/library/functions.html\#int}{\sphinxcode{int}}) \textendash{} duration of work in seconds

\item {} 
\sphinxstyleliteralstrong{from\_unixtime} (\sphinxhref{https://docs.python.org/2/library/functions.html\#int}{\sphinxcode{int}}) \textendash{} beginning of work in epoch

\item {} 
\sphinxstyleliteralstrong{to\_unixtime} (\sphinxhref{https://docs.python.org/2/library/functions.html\#int}{\sphinxcode{int}}) \textendash{} ending of work in epoch

\item {} 
\sphinxstyleliteralstrong{description} (\sphinxcode{Optional}{[}\sphinxhref{https://docs.python.org/2/library/functions.html\#str}{\sphinxcode{str}}{]}) \textendash{} description

\end{itemize}

\item[{Raises}] \leavevmode
\sphinxstyleliteralstrong{ConfusingData} \textendash{} when duration and from/to do are both given and do not match

\item[{Return type}] \leavevmode
{\hyperref[\detokenize{devmanual:protocol.Month}]{\sphinxcrossref{\sphinxcode{Month}}}}

\end{description}\end{quote}

\end{fulllineitems}

\index{append\_protocol() (protocol.Month method)}

\begin{fulllineitems}
\phantomsection\label{\detokenize{devmanual:protocol.Month.append_protocol}}\pysiglinewithargsret{\sphinxbfcode{append\_protocol}}{\emph{protocol}}{}
add a list or tuple of entries to the protocol
\begin{quote}\begin{description}
\item[{Return type}] \leavevmode
{\hyperref[\detokenize{devmanual:protocol.Month}]{\sphinxcrossref{\sphinxcode{Month}}}}

\end{description}\end{quote}

\end{fulllineitems}

\index{dump() (protocol.Month method)}

\begin{fulllineitems}
\phantomsection\label{\detokenize{devmanual:protocol.Month.dump}}\pysiglinewithargsret{\sphinxbfcode{dump}}{}{}
return a dict with all values
\begin{quote}\begin{description}
\item[{Return type}] \leavevmode
\sphinxhref{https://docs.python.org/2/library/stdtypes.html\#dict}{\sphinxcode{dict}}

\end{description}\end{quote}

\end{fulllineitems}

\index{get\_next() (protocol.Month method)}

\begin{fulllineitems}
\phantomsection\label{\detokenize{devmanual:protocol.Month.get_next}}\pysiglinewithargsret{\sphinxbfcode{get\_next}}{\emph{month=None}}{}
return the next Month
\begin{quote}\begin{description}
\item[{Return type}] \leavevmode
{\hyperref[\detokenize{devmanual:protocol.Month}]{\sphinxcrossref{\sphinxcode{Month}}}}

\end{description}\end{quote}

\end{fulllineitems}

\index{pretty() (protocol.Month method)}

\begin{fulllineitems}
\phantomsection\label{\detokenize{devmanual:protocol.Month.pretty}}\pysiglinewithargsret{\sphinxbfcode{pretty}}{}{}
return object as pretty string
\begin{quote}\begin{description}
\item[{Return type}] \leavevmode
\sphinxhref{https://docs.python.org/2/library/functions.html\#str}{\sphinxcode{str}}

\end{description}\end{quote}

\end{fulllineitems}


\end{fulllineitems}

\index{Season (class in protocol)}

\begin{fulllineitems}
\phantomsection\label{\detokenize{devmanual:protocol.Season}}\pysiglinewithargsret{\sphinxbfcode{class }\sphinxcode{protocol.}\sphinxbfcode{Season}}{\emph{holidays\_left=0}, \emph{working\_hours\_account=0}}{}
A {\hyperref[\detokenize{devmanual:protocol.Season}]{\sphinxcrossref{\sphinxcode{Season}}}} contains several Months in the same preset.

A preset is the set of monthly\_target, holidays and working\_hours\_account
Normally this should apply to a year
\index{add\_month() (protocol.Season method)}

\begin{fulllineitems}
\phantomsection\label{\detokenize{devmanual:protocol.Season.add_month}}\pysiglinewithargsret{\sphinxbfcode{add\_month}}{\emph{month}}{}
add a protocol to {\color{red}\bfseries{}:var:{}`self.protocols{}`}
:param protocol: protocol to add.
self.protocols{[}protocol.month{]} = protocol
\begin{quote}\begin{description}
\item[{Return type}] \leavevmode
{\hyperref[\detokenize{devmanual:protocol.Season}]{\sphinxcrossref{\sphinxcode{Season}}}}

\end{description}\end{quote}

\end{fulllineitems}

\index{validate() (protocol.Season method)}

\begin{fulllineitems}
\phantomsection\label{\detokenize{devmanual:protocol.Season.validate}}\pysiglinewithargsret{\sphinxbfcode{validate}}{}{}
validate the chain of months
\begin{quote}\begin{description}
\item[{Raises}] \leavevmode
\sphinxstyleliteralstrong{ValueError} \textendash{} when validation fails

\item[{Return type}] \leavevmode
{\hyperref[\detokenize{devmanual:protocol.Season}]{\sphinxcrossref{\sphinxcode{Season}}}}

\item[{Returns}] \leavevmode
self

\end{description}\end{quote}

\end{fulllineitems}


\end{fulllineitems}



\section{Changelog}
\label{\detokenize{changelog::doc}}\label{\detokenize{changelog:changelog}}
\DUrole{versionmodified}{New in version 0.1a: }Documentation


\chapter{Indices and tables}
\label{\detokenize{index:indices-and-tables}}\begin{itemize}
\item {} 
\DUrole{xref,std,std-ref}{genindex}

\item {} 
\DUrole{xref,std,std-ref}{modindex}

\item {} 
\DUrole{xref,std,std-ref}{search}

\end{itemize}


\renewcommand{\indexname}{Python Module Index}
\begin{sphinxtheindex}
\def\bigletter#1{{\Large\sffamily#1}\nopagebreak\vspace{1mm}}
\bigletter{p}
\item {\sphinxstyleindexentry{parser}}\sphinxstyleindexpageref{devmanual:\detokenize{module-parser}}
\item {\sphinxstyleindexentry{protocol}}\sphinxstyleindexpageref{devmanual:\detokenize{module-protocol}}
\end{sphinxtheindex}

\renewcommand{\indexname}{Index}
\printindex
\end{document}